\documentclass[aspectratio=169]{beamer}

% Must be loaded first
\usepackage{tikz}
\usetikzlibrary{shapes,arrows,positioning,fit,calc}

\usepackage[utf8]{inputenc}
\usepackage{textpos}

% Font configuration
\usepackage{fontspec}

\usepackage[listings, minted]{tcolorbox}
\usepackage{smartdiagram}
\usepackage[export]{adjustbox}

\input{font.tex}

% Tikz for beautiful drawings
\usetikzlibrary{mindmap,backgrounds}
\usetikzlibrary{arrows.meta,arrows}
\usetikzlibrary{shapes.geometric}

% Minted configuration for source code highlighting
\usepackage{minted}
\setminted{highlightcolor=orange!50, linenos}
\setminted{style=lovelace}

\usepackage[listings, minted]{tcolorbox}
\tcbset{left=6mm}

% Use the include theme
\usetheme{codecentric}

% Metadata
\title{The Power Of Const}
\author{Markus Hauck (@markus1189)}

\begin{document}

\begin{frame}[noframenumbering,plain]
  \titlepage{}
\end{frame}

\section{Introduction}\label{sec:introduction}

\begin{frame}
  \frametitle{Why This Talk}
  \begin{itemize}
  \item why a whole talk about \texttt{Const}?
  \item one of the ``funny'' datatypes at first sight
  \item not immediately obvious what it is good for
  \item surprisingly, it is \textbf{many} useful usages
  \item this talk shows (some of) them
  \end{itemize}
\end{frame}

\begin{frame}[fragile]
  \begin{center}
    \begin{minted}[fontsize=\LARGE, linenos=false]{haskell}
  newtype Const a b = Const { getConst :: a }
    \end{minted}
  \end{center}
\end{frame}

\begin{frame}[fragile]
  \frametitle{The Const Datatype}
  \begin{minted}{haskell}
    newtype Const a b = Const { getConst :: a }
  \end{minted}
  \vfill
  \begin{itemize}
  \item two type parameters \texttt{a} and \texttt{b}
  \item \texttt{b} is a phantom type
  \item you can only ever get a value of type \texttt{a} out of it
  \item type level version of the \texttt{const} \textbf{function} that ignores one of
    the two arguments:
  \end{itemize}
  \vspace{5mm}
  \begin{minted}{haskell}
    const :: a -> b -> b
    const x _ = x
  \end{minted}
\end{frame}

\begin{frame}
  \frametitle{The Rest Of This Talk}
  \begin{itemize}
  \item \texttt{Functor} instance
  \item \texttt{Applicative} instance
  \item \texttt{Selective Applicative} instance
  \end{itemize}
\end{frame}

\section{Functor}
\begin{frame}
  \begin{center}
    \Huge Functor
  \end{center}
\end{frame}

\begin{frame}[fragile,t]
  \frametitle{The Functor Instance}
  \begin{minted}{haskell}
  instance Functor (Const a) where
    fmap f c@(Const a) = _
  \end{minted}
  \vspace{5mm}
  \begin{minted}{text}
  • Found hole: _ :: MyConst a b
  ...
  • Relevant bindings include
      c :: MyConst a a1
      a :: a
      f :: a1 -> b
  \end{minted}
\end{frame}

\begin{frame}[fragile,t]
  \frametitle{The Functor Instance}
  \begin{minted}{haskell}
  instance Functor (Const a) where
    fmap _ (Const a) = Const a
  \end{minted}
  \vspace{5mm}
  \begin{minted}{text}
  ghci> c = Const @String @Int "can't touch me"

  ghci> fmap (+1) c
  Const "can't touch me"

  ghci> fmap print c
  Const "can't touch me"
  \end{minted}
  \vfill
  \begin{itemize}
  \item unpack and retag (change the phantom type)
  \item \textbf{discards} the function application
  \item \dots{} but how is this useful?
  \end{itemize}
\end{frame}

\begin{frame}[fragile]
  \frametitle{Use of Const in Twan van Laarhoven Lenses}
  \begin{minted}{haskell}
    type Lens a b = forall f. Functor f => (b -> f b) -> a -> f a
    _1 :: Lens (a, b) a
    _1 f (x1, y) = fmap (\x2 -> (x2, y)) (f x1)
  \end{minted}
  \vfill
  \begin{itemize}
  \item think: '\textbf{\texttt{a}}' somehow contains '\textbf{\texttt{b}}'
  \item given:
    \begin{itemize}
    \item functorial function modifying the \textbf{\texttt{b}}
    \item and a ``bigger'' value of type \textbf{\texttt{a}}
    \end{itemize}
  \item produce new \textbf{\texttt{a}} that has the modified \textbf{\texttt{b}} ``inside''
  \end{itemize}
\end{frame}

\begin{frame}[fragile]
  \frametitle{Implementing Getter With Const}
  \begin{minted}{haskell}
    _1 :: Lens (a, b) a
    _1 f (x1, y) = fmap (\x2 -> (x2, y)) (f x1)

    get :: Lens a b -> a -> b
    get l x = getConst (l Const x)
  \end{minted}
  \vfill
  \begin{minted}{haskell}
  get _1 (42, 'a')
= getConst (_1 Const (42, 'a'))                 -- Def. of 'get'
= getConst (fmap (\x2 -> (x2, 'a')) (Const 42)) -- Def. of '_1'
= getConst (Const 42)                           -- Def. of 'fmap' for 'Const
= 42                                            -- The Answer!
  \end{minted}
\end{frame}

\begin{frame}
  \frametitle{Functor}
  \begin{itemize}
  \item first example: Functor for \texttt{Const}
  \item useless at first glance
  \item nice for lenses
  \end{itemize}
\end{frame}

\section{Applicative}
\begin{frame}
  \begin{center}
    \Huge Applicative
  \end{center}
\end{frame}


\begin{frame}
  \frametitle{The Applicative Instance}
  \begin{itemize}
  \item probably one of my favorites
  \item opens up a lot of possibilities
  \item how? connects two very important concepts
  \end{itemize}
\end{frame}

\begin{frame}[fragile]
  \frametitle{The Applicative Instance}
  \begin{minted}{haskell}
    instance Applicative (Const m) where
      pure :: a -> Const m a
      pure = _pure

      (<*>) :: Const m (a -> b) -> Const m a -> Const m b
      Const f <*> Const v = _ap
  \end{minted}
\end{frame}

\begin{frame}[fragile]
  \frametitle{The Applicative Instance}
  \begin{minted}{haskell}
    instance Monoid m => Applicative (Const m) where
      pure :: a -> Const m a
      pure _ = Const mempty

      (<*>) :: Const m (a -> b) -> Const m a -> Const m b
      Const f <*> Const v = Const (f <> v)
  \end{minted}
\end{frame}

\begin{frame}
  \frametitle{Using The Applicative Instance}
  \begin{itemize}
  \item but what does it \textbf{mean}?
  \item we can use any function working with \texttt{Applicative}s and
    give them \texttt{Monoid}s!
  \end{itemize}
\end{frame}

\begin{frame}[fragile,t]
  \frametitle{Using The Applicative Instance}
  \begin{minted}{haskell}
traverse :: ... => (a -> Const m b) -> t a -> Const m (t b)
foldMap  :: ... => (a ->       m)   -> t a ->       m
  \end{minted}
\vfill
  \begin{enumerate}
  \item
    \begin{itemize}
    \item by using \texttt{traverse} with \texttt{Const} we get \texttt{foldMap}
    \item that's why \texttt{Traversable} is enough to define a \texttt{Foldable} instance
    \end{itemize}
  \item<2->
    \begin{itemize}
    \item Use \texttt{Const} to statically analyze \texttt{Free Applicative} programs
    \item to accumulate monoidal value without performing actual effects
    \end{itemize}
  \item<3->
    \begin{itemize}
    \item \texttt{Const} highlights the relation between
      Applicative and Monoid
    \item use \textbf{everything} from Monoids and use with functions that require \texttt{Applicative}
    \item Applicative laws state that instances have to be monoidal in their effects
    \end{itemize}
  \end{enumerate}
\end{frame}

\begin{frame}[fragile]
  \frametitle{What Is Next?}
  \begin{itemize}
  \item we saw the instance for \texttt{Applicative}
  \item important bridge between \texttt{Monoid} and \texttt{Applicative}
  \item so what about the \texttt{Monad} instance?
  \end{itemize}
\end{frame}

\begin{frame}[fragile]
  \frametitle{The Monad Instance?}
  \begin{minted}{haskell}
import Data.Functor.Const

instance Monoid a => Monad (Const a) where
  c@(Const x) >>= f = f _
  \end{minted}
  \begin{minted}{text}
const-monad.hs:4:23: error:
    • Found hole: _ :: Const a b
...
    • Relevant bindings include
        f :: a1 -> Const a b (bound at const-monad.hs:4:19)
        x :: a (bound at const-monad.hs:4:12)
        c :: Const a a1 (bound at const-monad.hs:4:3)
...
  \end{minted}
\end{frame}

\begin{frame}[fragile]
  \begin{itemize}
  \item \texttt{mempty} would typecheck, but violate the left-identity law
  \item we simply can't ``pretend'' anymore
  \item \texttt{Const} does not have a \texttt{Monad} instance, but
    still manages to teach us
  \item somewhere between \texttt{Applicative} and
    \texttt{Monad}\ldots
  \end{itemize}
\end{frame}

\section{Selective}
\begin{frame}
  \begin{center}
    \Huge Selective Applicative Functors
  \end{center}
\end{frame}

\begin{frame}[fragile]
  \begin{center}
    \includegraphics[width=0.76\textwidth]{static-images/selective-applicative-functors-paper.jpg}
    \vfill
    \href{https://www.staff.ncl.ac.uk/andrey.mokhov/selective-functors.pdf}{https://www.staff.ncl.ac.uk/andrey.mokhov/selective-functors.pdf}
  \end{center}
\end{frame}

\begin{frame}
    \begin{quote}
    This paper introduces an intermediate abstraction called
    \textit{selective applicative functors} that requires all effects
    to be \textbf{declared statically}, but provides a way to select which of
    the effects to \textbf{execute dynamically}.
  \end{quote}

  \begin{itemize}
  \item (emphasis mine)
  \item \texttt{Applicative}: effects declared \& executed statically
  \item offer some of the benefits of Arrows, less powerful (not this talk \texttt{:/})
  \end{itemize}
\end{frame}

\begin{frame}[fragile]
  \frametitle{Selective Functors}
  \begin{minted}{haskell}
class Applicative f => Selective f where
  select :: f (Either a b) -> f (a -> b) -> f b
  \end{minted}
  \vspace{5mm}
  \begin{itemize}
  \item (comes with some additional laws not shown here)
  \item what does it buy me?
  \item you can branch on \textbf{Bool}s that are inside an ``effect''\footnote{``effect'' with the usual caveats}
  \end{itemize}
  \begin{minted}{haskell}
    ifS :: Selective f => f Bool -> f a -> f a -> f a
  \end{minted}
\end{frame}

\begin{frame}[fragile,t]
  \frametitle{The Selective Instance}
    \begin{itemize}
  \item now the interesting part: how does the instance for \texttt{Const} work?
  \item with \texttt{Selective} we have two valid instances
  \end{itemize}
  \vfill
  \begin{minted}{haskell}
newtype Over m a = Over { getOver :: m }

newtype Under m a = Under { getUnder :: m }
  \end{minted}
  \vfill
  \begin{itemize}
  \item \texttt{Over} for static \textbf{over}-approximation
  \item \texttt{Under} for static \textbf{under}-approximation
  \end{itemize}
\end{frame}

\begin{frame}[fragile]
  \begin{minted}{haskell}
instance Monoid a => Selective (Over a) where
  select (Over (Const c)) (Over (Const t)) = Over (Const (c <> t))

instance Monoid a => Selective (Under a) where
  select (Under (Const c)) _ = Under (Const c)
  \end{minted}
  \vfill
  \begin{itemize}
  \item \textbf{\texttt{Over}} also goes into conditional branches
  \item \textbf{\texttt{Under}} ignores conditionally executed parts
  \end{itemize}
\end{frame}

\begin{frame}[fragile]
  \begin{itemize}
  \item why is this so cool?
  \item DSL using \textbf{\texttt{FreeSelective}} can express conditional branching and do
    static analysis/transformation on programs
  \item using \texttt{Over} and \texttt{Under} extremely useful to e.g.\ check properties of a program
  \item example: does a certain effect \textbf{always} occur? Check \texttt{Under}
  \item example: does a certain effect \textbf{never} occur: Check \texttt{Over}
  \end{itemize}
\end{frame}

\section{Conclusion}
\begin{frame}
  \begin{center}
    \Huge Conclusion
  \end{center}
\end{frame}

\begin{frame}
  \begin{itemize}
  \item deceptively easy data type
  \item very useful instances that have surprising usages:
    \begin{itemize}
    \item Functor
    \item Applicative
    \item Selective Applicative
    \end{itemize}
  \item good tool to understand Monoidal aspects of Applicatives
  \item highlights the difference between monadic and applicative
    capabilities
  \end{itemize}
\end{frame}

\begin{frame}
  \begin{center}
    \Huge{}
    Questions?
  \end{center}
\end{frame}

\appendix{}

\end{document}
